\section{HBASE}
\begin{displayquote}
    \textit{\textbf{CAP:} CP - Konsistent med en høj partitions tolerance}
\end{displayquote}

HBase er en kolonneorienteret database, som er bygget ovenpå Hadoops File Systems(HDFS). Det er et open-source projekt og da der er tale om en nosql database, er det muligt at skalere databasen horisontalt. HBases datamodel er baseret på Google’s “big table”, der er designet til at give en hurtig adgang til store mængder struktureret data og samtidig udnytte HBase HDFS’ fejltolerance. Desuden er HBase en del af Hadoops ecosystem, der udbyder real-time adgang til læsning og skrivning af data til HDFS. Det er muligt at gemme data direkte i HDFS eller gennem HBase, men for at kunne læse data fra HDFS, skal det gå igennem HBase. I vores projekt har vi valgt at bruge HBase til at gemme den data, der beskriver, hvor langt en bruger er kommet med en given film eller serie.
\begin{tcolorbox}
    \lstset{style=htmlstyle}
    \begin{lstlisting}[language={[Sharp]C}, caption={Watchlist HBASE Model}, label={lst:watchlist}]
        watchlist
            <profile id>
                <movie/series id>:type = <type>
                <movie/series id>:timestamp = <timestamp>
                <movie/series id>:season_id = <season id>
                <movie/series id>:episode_id = <episode id>

        watchlist
            1234
                5:type = movie
                5:timestamp = 1600
                2:type = series
                2:timestamp = 1200
                2:season_id = 3
                2:episode_id = 2
            1235
                5:type = movie
                5:timestamp = 45000
                4:type = movie
                4:timestamp = 200
    \end{lstlisting}
\end{tcolorbox}
Det fremgår på listing \ref{lst:watchlist}, at vi har en en overordnet tabel kaldet watchlist. I denne tabel gemmer vi profil id’et som vores række nøgle efterfulgt af id’et på den pågældende film eller serie. Derudover gemmer vi også, hvilken type materiale der er tale om - altså film eller serie - samt et timestamp i sekunder på, hvor langt denne profil er kommet med den givne film eller serie. Hvis der er tale om en serie gemmer vi samtidig id’et på afsnittet, der er tale om og id’et på, hvilken sæson vi finder afsnittet i. På listing \ref{lst:watchlist} fremgår der også et eksempel på en profil med id 1234, hvor det tydeligt ses at denne profil har set noget af en film med id 5 og har timestampet 1600 sekunder. Det fremgår også at samme profil er i gang med at se en serie, der har id 2, hvor der er tale om det 2. afsnit i den 3. sæson og afsnittet har så et timestamp på 1200 sekunder. På listing \ref{lst:watchlist} ses ydermere endnu en profil, der er i gang med at se to film med hhv. id 4 og 5.
\bigbreak

Udover at gemme en “liste” i HBase over, hvor langt hver profil er kommet med diverse film og serier, gemmer vi også en “liste” af logs.
\begin{tcolorbox}
    \lstset{style=htmlstyle}
    \begin{lstlisting}[language={[Sharp]C}, caption={Logs HBASE Model}, label={lst:log}]
        log
            <http_method>
                <path>:<epoc_timestamp> = <data>
    
        log
            GET
                /api/Account/get:1621847222436 = {Id: 1}
                /api/Account/get:1621847222435 = {Id: 5}
                /api/Account/get/all:1621847222436 = {}
            POST
                /api/Account/create:1621847222478 = {JSON DATA}
                /api/Account/create:1621847222578 = {JSON DATA}
                /api/Account/create:1621847222443 = {JSON DATA}
    \end{lstlisting}
\end{tcolorbox}
På listing \ref{lst:log} ser vi en tabel med navnet log, der har en http-metode som række nøgle efterfulgt af vejen, der er brugt til http-metoden som familie id samt et timestamp skrevet som unix timestamp med tilhørende json data. På listing \ref{lst:log} fremgår der også to eksempler af tabellen log, hvor der i det første eksempel er blevet kaldet en get metode på /api/Account/get med et tilhørende unix timestamp og json data (i dette tilfælde er json dataen det id på den film eller serie, brugeren ønsker at se). Det andet eksempel er en række post metoder, som skal symbolisere brugeroprettelser, hvor json dataen indeholder de nødvendige oplysninger for at oprette en bruger.