\section{Redis}
\begin{displayquote}
    \textit{\textbf{CAP:} CP - Konsistent med en høj partitions tolerance}
\end{displayquote}

Redis er en open-source, in-memory, key-value database. Redis understøtter forskellige abstrakte datastrukturer, men i vores tilfælde gør vi kun brug af strenge. Det er oplagt at bruge Redis til at håndtere meget data, da det er en in-memory database. Hvis vi tænker på en bruger, der gerne vil se en specifik film på Netflix, kræver det ikke så meget for databasen at læse det data, der tilhører den specifikke film. Hvis vi så i stedet tænker på at en bruger gerne vil se en liste af alle film, kræver det mere at læse al den data, der tilhører alle film og hvis vi så i stedet tænker på 1.000.000 brugere gerne vil læse al den data, der tilhører alle film, begynder det at tage tid. I dette projekt har vi derfor valgt at bruge Redis til at løse lige præcis dette problem.
\begin{tcolorbox}
    \lstset{style=htmlstyle}
    \begin{lstlisting}[language={[Sharp]C}, caption={Logs HBASE Model}, label={lst:cache}]
        cache:<cache_type>#<genre ifexists> = <json data>

        cache:Movie = {JSON DATA}
        cache:MovieByGenre#Fantasy = {JSON DATA}
        cache:Series = {JSON DATA}
        cache:SeriesByGenre#Fantasy = {JSON DATA}
    \end{lstlisting}
\end{tcolorbox}
På listing \ref{lst:cache} fremgår det at vi i Redis bruger \textit{cache:cache typen} og evt. en genre, hvis denne findes som vores nøgle og json data, som vores værdi. Json data er i dette tilfælde al den data, der er tilhørende de film eller serier, som brugeren beder om at få en liste af. På listing \ref{lst:cache} fremgår der også et par eksempler så, hvis vi fx. ser på nøglen \textit{cache:Movie} vil vi her få alt data, der er tilhørende alle film. Vi kan fx. også se på nøglen \textit{cache:MovieByGenre\#Fantasy} her får vi alt den data, der er tilhørende alle film med genren fantasy. På denne måde bruger vi Redis til at aflaste vores Neo4j servere. Det vil altså også sige, at vi i Redis gemmer flere strenge, eftersom en bruger kan bede om al data på både film, serier og disse kategoriseret efter deres genre. I vores tilfælde bliver der oprettet en streng igennem Redis, første gang en bruger kalder API’et for at få al data, der tilhører alle film. Hvis mere end en bruger ønsker at få al data, der tilhører alle film, vil strengen i Redis blive opdateret, hver gang en ny bruger modtager et svar. Grunden til dette er, at der ikke eksisterer en streng, der indeholder al data tilhørende alle film, før den første bruger har fået svar og derfor kan vi først returnere den pågældende streng, når en bruger ønsker at se al data, der tilhører alle film, når strengen faktisk eksisterer. Desuden opdaterer Redis hhv. den streng, der indeholder alle film og den streng, der indeholder alle film af en bestemt genre, hver gang der bliver oprettet en ny film med en bestemt genre. Det samme sker for serier.